%Westminster theme for beamer
%I have created this theme template for beamer to use to prepare for my lecture notes in Latex.
% Please feel free to reuse this template in anyway you need. 
% If there is any mistakes or anything you can fix, please let me know or you can commit on github.
%Ayman El Hajjar 
%31 August 2017

% Note, the presentation is designed with 169 aspect ratio. you can change this to 43 but you will have to change position of some of the tables and figures
\documentclass[aspectratio=169,xcolor=table]{beamer}
\usetheme{westminster}
\usepackage{graphicx}
\graphicspath{{./images/}}
\usepackage{tikz}
\usepackage[T1]{fontenc}
\usebeamerfont{professionalfonts}

\usepackage[absolute,overlay]{textpos}
\usepackage{extrabeamercmds}
%Lecture Name
\title{\textbf{My title}}
%Sub lecture name - If needed
\subtitle{\textbf{subtitle}}
% Author is the module leader 
\author[Module leader]{\textbf{Module Leader: Module leader}} %<= used the short author name [] for the footline

\addtobeamertemplate{navigation symbols}{}{%
	\usebeamerfont{footline}%
	\usebeamercolor[fg]{white}%


	\Large 	\insertshortauthor
		\hspace{13em}% 
		\Large\insertframenumber of \inserttotalframenumber
	
	
}
\begin{document}
	
	\begin{frame}
		\titlepage
	\end{frame}
	
	
	\begin{frame} 
		\frametitle{Frame 1: Theorems} 
		%frame subtitle can be left blank
		\framesubtitle{The proof uses \textit{reductio ad absurdum}.} 
				\begin{theorem}
					You can insert an equation .\\
					\begin{equation}
					E=mc^2
					\end{equation} \end{theorem} 
						\begin{theorem}
							You can insert an equation without equation number\\
		\begin{equation*}
							E=mc^2
		\end{equation*} \end{theorem} 
	
	\end{frame}
	
	\begin{frame}{Itemize and Enumerate}
		\begin{itemize}
			\item one
			\item two
			\item three
	
			\end{itemize}
			
					\begin{enumerate}
						\item one
						\item two
						\item three

					\end{enumerate}
	\end{frame}
	
	\begin{frame}{Two columns}
		\begin{columns}
				\begin{column}{4cm}
					\begin{itemize}
					\item one
					\item two
					\item three
					\item four
					\item five
					\end{itemize}
				\end{column}

				\begin{column}{4cm}
					\begin{itemize}
					\item one
					\item two
					\item three
					\item four
					\item five
					\end{itemize}
				\end{column}
		\end{columns}
	\end{frame}
		
\begin{frame}{Adding video}
	
	\begin{textblock*}{4cm}(4cm,2cm) % {block width} (coords)
	\href{run:images/video.mp4?autostart&loop}{\includegraphics[width=6cm,height=6cm]{video.jpg}}
	\end{textblock*}

\end{frame}

		
		\begin{frame}{Adding images at random positions with animation}
			
	\begin{tikzpicture}[remember picture,overlay]
	\only<1>{\node[xshift=4.2cm,yshift=4.5cm] at (current page.south west) {\includegraphics[width=3cm, height=5cm]{image.png}};}
	\only<2>{\node[xshift=8.2cm,yshift=4.5cm] at (current page.south west) {\includegraphics[width=3cm, height=5cm]{image.png}};}
	\only<3>{\node[xshift=12.2cm,yshift=3.5cm] at (current page.south west) {\includegraphics[width=3cm, height=5cm]{image.png}};}
	\end{tikzpicture}
			
		\end{frame}
	
\begin{frame}{Question and answers}
	\begin{itemize} 
		\item<1-| alert@1> Question 1  text
		\begin{itemize}
		\item<2->  Answer 1
		\item<2-> Answer 1
		\end{itemize}
		\item<1-| alert@1> Question 2  text
		\begin{itemize}
			\item<3->  Answer 2
			\item<3-> Answer 2
		\end{itemize}
	\end{itemize}
\end{frame}	

\begin{frame}{Itemize \& pause}
						\begin{itemize}
							\item one
							\pause
							\item two
							\pause
							\item three
														\pause
							\item four
														\pause
							\item five
						\end{itemize}
	\end{frame}	
	
	\begin{frame}{Blocks}
\begin{block}{test}
Reference to image in Fig.\ref{imagelable}
\end{block}
	\end{frame}	
	
		\begin{frame}{Blocks}
		\begin{figure}
		\includegraphics[width=10cm, height=4cm]{image}
		  \caption{Image caption}
		  \label{imagelable}
		\end{figure}
		\end{frame}	
		
		
			
			\begin{frame}{How to work tables}
				\begin{columns}
					\begin{column}{5cm}
\begin{tabular}{|c|c|c|c|c|c|c|}
	\hline
	\cellcolor{universitycolor}A&\cellcolor{grayy}B&\cellcolor{universitycolor}C&\cellcolor{grayy}B&\cellcolor{universitycolor}A&\cellcolor{grayy}B&\cellcolor{universitycolor}C\\
	\hline
	\hline
	\cellcolor{universitycolor}A&\cellcolor{grayy}B&\cellcolor{universitycolor}C&\cellcolor{grayy}B&\cellcolor{universitycolor}A&\cellcolor{grayy}B&\cellcolor{universitycolor}C\\
	\hline
	\hline
	\cellcolor{universitycolor}A&\cellcolor{grayy}B&\cellcolor{universitycolor}C&\cellcolor{grayy}B&\cellcolor{universitycolor}A&\cellcolor{grayy}B&\cellcolor{universitycolor}C\\
	\hline
	\hline
	\cellcolor{universitycolor}A&\cellcolor{grayy}B&\cellcolor{universitycolor}C&\cellcolor{grayy}B&\cellcolor{universitycolor}A&\cellcolor{grayy}B&\cellcolor{universitycolor}C\\
	\hline
\end{tabular}
					\end{column}
					\begin{column}{5cm}
					Test 
					\end{column}
					
			\end{columns}

			\end{frame}
			
\end{document}